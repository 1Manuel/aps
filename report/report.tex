\documentclass[parskip=full]{scrartcl}
\usepackage[utf8]{inputenc} % use utf8 file encoding for TeX sources
\usepackage[T1]{fontenc}    % avoid garbled Unicode text in pdf
\usepackage[german]{babel}  % german hyphenation, quotes, etc
\usepackage{hyperref}       % detailed hyperlink/pdf configuration
\hypersetup{                % ‘texdoc hyperref‘ for options
pdftitle={APS: Report for APP2},%
bookmarks=true,%
}
\usepackage{graphicx}       % provides commands for including figures
\usepackage{csquotes}       % provides \enquote{} macro for "quotes"
\usepackage[nonumberlist]{glossaries}     % provides glossary commands
\usepackage{enumitem}

\setlength{\emergencystretch}{3em}


\makenoidxglossaries

\newglossaryentry{Entry_name}
{
	name=dummy,
	plural=dummies,
	description={description},
}


\title{APP2: Report}
\author{Sofiane TANJI, Habib SLIM, Eslam MOHAMMED,\\ Archit YADAV, Albert STRÜMPLER, Manuel TREUTLEIN}
\begin{document}

\maketitle

\section{Introduction}
-- Our first thoughts on the problem (session 1)

\section{The greedy algorithm}
-- Section 2 basically

\subsection{Pseudocode}
-- write down the pseudocode

\subsection{Complexity}
-- short chapter about the complexity, maybe not necessary.

\subsection{Limitations and advantages}
-- maybe make a table


\section{Complete solution space exploration}
-- Section 3 basically

\subsection{Pseudocode}
-- write down the pseudocode. 

\subsection{Complexity}
-- short chapter about the complexity.

\subsection{Discussion about possible filters.}
-- short chapter about the complexity.

\printnoidxglossaries

\end{document}
