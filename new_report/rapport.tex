% !TEX encoding = UTF-8 Unicode
% -*- coding: UTF-8; -*-
% vim: set fenc=utf-8
\documentclass[a4paper,12pt]{article}

\usepackage{multirow}
\usepackage[table,xcdraw]{xcolor}
\usepackage{centrale}
\usepackage{minted}
\usepackage{enumitem}
\usepackage{graphicx}       % provides commands for including figures
\usepackage{csquotes}       % provides \enquote{} macro for "quotes"
\usepackage{glossaries}     % provides glossary commands
\lstset{
    frame=tb, % draw a frame at the top and bottom of the code block
    tabsize=4, % tab space width
    showstringspaces=false, % don't mark spaces in strings
    numbers=left, % display line numbers on the left
    commentstyle=\color{green}, % comment color
    keywordstyle=\color{blue}, % keyword color
    stringstyle=\color{red} % string color
}

\hypersetup{
    pdftitle={APP2 Report},
    pdfauthor={Habib Slim, Sofiane Tanji, Eslam Mohammed, Archit Yadav, Albert Strümpler, Manuel Treutlein},
    pdfsubject={Problem Solving},
    pdfproducer={},
    pdfkeywords={greedy, dynamic programming, complete solution space exploration, holdem, cards} %
}

\DeclareGraphicsRule{.ai}{pdf}{.ai}{} % pour insérer des documents .ai
\graphicspath{ {./img/} {./eps/}} % pour ne pas avoir à ajouter eps/ton-image.jpg

% ------------- Packages spéciaux, nécessaires pour ce rapport, à insérer ici ------------- 

\makeglossaries
\newglossaryentry{deque} {
	name = double ended linked list,
	description = {Standard data structure in algorithms. Each element in the list
								 is linked by a pointer to the left and a pointer to the right element, exept
								 the first and the last element. We can access the list only through a pointer
								 on the first and last element.}
}
\newglossaryentry{pop} {
	name = pop(),
	description = {Get the last element in the deque. In terms of the card game this
								 is the rightmost card.}
}
\newglossaryentry{popleft} {
	name = popleft(),
	description = {Get the first element in the deque. In terms of the card game this
								 is the leftmost card.}
}
\newglossaryentry{greedy} {
	name = greedy algorithm,
	description = {The greedy algorithm is a heuristic algorithm pattern to solve
								 problems through making optimal local choices. This might not
								 be optimal in the global context.}
}

\begin{document}

% --------------------------------------------------------------
%                       Page de garde
% --------------------------------------------------------------

\begin{titlepage}
\begin{center}

\includegraphics[width=0.35\textwidth]{logo-uga.png}
\includegraphics[width=0.35\textwidth]{logoinp.png} \\[1cm]

{\large Master M1 MOSIG – UGA \& Grenoble INP} \\[0.8cm]
{\large Algorithmic Problem Solving}\\[0.5cm]

% Title
\rule{\linewidth}{0.5mm} \\[0.4cm]
{ \huge \bfseries APP2: Hold’em for n00bs \\[0.4cm] }
\rule{\linewidth}{0.5mm} \\[1.5cm]

% Author and supervisor
\noindent
\begin{minipage}{0.4\textwidth}
  \begin{flushleft} \large
    \emph{Authors :}\\
    Habib \textsc{Slim}\\
    Eslam \textsc{Mohammed}\\
    Archit \textsc{Yadav}\\
    Albert \textsc{Strümpler}
    Manuel \textsc{Treutlein}\\
    Sofiane \textsc{Tanji}
  \end{flushleft}
\end{minipage}%
\begin{minipage}{0.4\textwidth}
  \begin{flushright} \large
    \emph{Teacher :} \\
    Ms.~Malin \textsc{Rau}\\
  \end{flushright}
\end{minipage}

\vfill

% Bottom of the page
{\large Last Version \\ \today}

\end{center}
\end{titlepage}

% --------------------------------------------------------------
%                    Table des matières 
% --------------------------------------------------------------

\thispagestyle{empty}
\tableofcontents
\newpage
% --------------------------------------------------------------
%                         Début du corps
% --------------------------------------------------------------
\section{Introduction and modelling}
% Explanation of the problem and modelling.
In this APP we have to deal with a card game played by two persons.
One player is referred to as sister, the other player is referred to as strategist.
A series of n cards lying on the table face up in a line. We are modelling
the card deck through a \gls{deque}, short deque. The elements of the deque are integers
in the range of [2, 14], whereas the value 2 represents the card 2 and the
value 14 represents the value of an ace. All values in between are assigned
appropriately, this means in particular for the face cards that the jack
is modeled by 11, the queen by 12 and the king by 13.
The deque allows according to the game rules to take only the
rightmost or the leftmost card. The players take turns while playing the
game and always decide to take the leftmost or rightmost card. This is
modeled through \gls{popleft} to take leftmost card and through \gls{pop}
to take the rightmost card. Be careful, the naming of \gls{pop} for taking
the rightmost card can be considered not consistent with \gls{popleft}, but we
stick to the denotation of the phyton standard library. The player with
the highest score in the end wins. Even though we will apply different
algorithms to solve the problem, the input and ouput of the algorithm stays the same.
%% TODO change this section if input or ouput should be different
The input is a list of cards represented as deque and the choice (of the sister)
who starts the game represented as boolean. The output is 0 if the sister wins
or the sum of the players are equal. The output is 1 if the
player strategist wins.\\

In this APP we decided to change to python code for representing algorithms.
The reason for this is that python code has a simple structure and resembles
pseudo-code in a way. But furthermore it allows us to represent the developed
algorithms more detailed.\\

The sister will always play the so called \gls{greedy}. Therefore we first consider an algorithm applying this method in chapter 2. Because of some limitations of the \gls{greedy} we will then consider a complete solution space exploration in chapter 3. This gives us an optimal solution, but with an unacceptable runtime. For this reason we will introduce a dynamic algorithm in chapter 4, resulting in an optimal solution with acceptable runtime.

% --------------------------------------------------------------
%                         Partie 1
% --------------------------------------------------------------

\newpage

\section{The greedy algorithm}
%% TODO complete, RESPONSIBLE PERSON: Albert
-- Section 2 basically

\subsection{Pseudocode}
%% TODO complete, RESPONSIBLE PERSON: Sofiane
-- write down the python pseudocode

\subsection{Complexity}
%% TODO complete, RESPONSIBLE PERSON: Sofiane
-- short chapter about the complexity, maybe not necessary.

\subsection{Limitations and advantages}
%% TODO complete, RESPONSIBLE PERSON: Manuel
-- maybe make a table


% --------------------------------------------------------------
%                         Partie 2
% --------------------------------------------------------------

\newpage

\section{Complete solution space exploration}
%% TODO complete, RESPONSIBLE PERSON: Albert
-- Section 3 basically

\subsection{Pseudocode}
%% TODO complete, RESPONSIBLE PERSON: Habib, Jimmy
-- write down the python pseudocode

\subsection{Complexity}
%% TODO complete, RESPONSIBLE PERSON: Achret
-- short chapter about the complexity.

%% \subsection{Discussion about possible filters.}
%% TODO ONLY if we have time.
%% TODO complete, RESPONSIBLE PERSON:
%% -- short chapter about the complexity.


% --------------------------------------------------------------
%                         Partie 2
% --------------------------------------------------------------

\newpage


\section{Solution with dynamic programming}
%% TODO complete, RESPONSIBLE PERSON: Jimmy
-- Section 4 basically


\subsection{Pseudocode}
%% TODO complete, RESPONSIBLE PERSON: Jimmy, Habib
For the algorithm we make some assumptions to make the code
easier to understand. This leads to the necessary to adapt
the code under after assumptions. We will consider these later.
The assumptions
\begin{itemize}
	\item The player strategist always starts.
	\item There can never occur the situation that the sister chooses greedy between two cards with the same value.
\end{itemize}

-- write down the python pseudocode



\subsection{Complexity}
%% TODO complete, RESPONSIBLE PERSON: Jimmy, Habib
-- short chapter about the complexity.

\subsection{The algorithm under different assumptions}
% TODO complete, RESPONSIBLE PERSON: Habib
-- add the consideration of different assumptions for the dynamic programming.


% --------------------------------------------------------------
%                         Conclusion
% --------------------------------------------------------------

\section{Conclusion and feedback}

%% TODO complete, RESPONSIBLE PERSON: Manuel
-- Write something in general

% --------------------------------------------------------------
%                            Abstract
% --------------------------------------------------------------

\end{document}
